\documentclass[ps1.tex]{subfiles}
%\usepackage{enumerate}
%\usepackage[left=3cm,right=3cm,bottom=3cm,top=2cm]{geometry}
\begin{document}

\section* {(25 points) Dimensional Analysis: Two Kinds of Quantum Gravity}
\begin{enumerate}[(a)]
\item \underline{Gravitational bound states}
\noindent
Consider a particle sitting on a table which is kept from floating away only by the
force of gravity. This system is characterized by just three physical parameters,
the mass of the particle, m, the acceleration of gravity on Earth, g = 9.8 $\frac {m}{s^2}$ , and
Planck's constant, $\hbar = \frac {1}{2\pi}h$. The energy given by $E = 12 mv^2 + mgx$.
\begin{enumerate}[i.]
\item Using only dimensional analysis, find the product of powers of m, g, $\hbar$ which give a characteristic energy, E. (i.e., write $E =  m^{\alpha}g^{\beta}\hbar^{\gamma}$ and solve for $\alpha, \beta, \gamma)$ Can you find such a characteristic energy without using the Planck constant?
\item Repeat to find characteristic length, time, and speeds (l, t, v) for this system
\item Classically, putting the system in its lowest energy configuration $(E=0)$ would require 
the particle to sit perfectly still $(v = 0)$ precisely on the surface $(x = 0)$. 
Use the uncertainty relation, $\delta x \delta p \ge \frac {\hbar}{2}$ , to argue (briefly!) that the particle cannot have $E = 0$ while respecting the uncertainty principle. \\
ASIDE: Quantum mechanically, then, there must be some minimum energy this system can have which cannot be predicted classically! For a particle on a table, this may not seem so important - but for Hydrogen, which you've just shown to be classically unstable, this is absolutely key. We will soon learn how to calculate the minimum ("ground state") energy of such systems.
\item Use your dimensional analysis results to give a simple estimate for the ground state energy of this system. How does your estimate behave as $h \rightarrow 0$? Does this make sense? Explain why or why not.
\item Evaluate $E, l, t $ and v numerically for a neutron $(m_N = 1.7*10^{27}kg)$. How high above the surface will the particle typically be found?
\end{enumerate}

\item \underline{The Planck Scale}
\noindent
The scale at which gravity (characterized by the Newton constant, $G_N$), 
quantum mechanics ($\hbar$), and relativity (c) are all important is called the Planck scale.
\begin{enumerate}[i.]
\item Using dimensional analysis, find the combination of powers of $G_N, \hbar$ and $c$ which make a length - we call this the Planck length, $L_P$ .
\item Evaluate $L_P$  numerically, and compare to a typical scale for nuclear or particle physics, namely $1F = 10^{-15}$ m.
\item Repeat to find the Planck mass, $M_P$ , evaluate it numerically, and compare to the mass of a typical nuclear constituent (like the proton mass). Do we need to understand Quantum Gravity to study nuclear physics?
\end{enumerate}
\end{enumerate}
\noindent\makebox[\linewidth]{\rule{\paperwidth}{0.4pt}}

\begin{enumerate}[(a)]
\item \mbox{} \\ 
\noindent
\begin{enumerate}[i.]
\item $E \approx m^{\alpha}g^{\beta}\hbar^{\gamma}$

$[kg]^1[m]^2[s]^{-2} = [kg]^{\alpha}[\frac {m}{s^2}]^{\beta}[\frac {kgm^2}{s}]^{\gamma}$

so set up equations of exponents

$kg  \rightarrow 1 = 1\alpha + 0\alpha + 1\alpha$\\
$m  \rightarrow 2 = 0\beta + 1\beta + 2\beta$\\
$sec  \rightarrow 1 = 1\gamma + 0\gamma + 1\gamma$\\

solve simultaneous equations and this gives 
$\alpha = \frac {1}{3}, \beta = \frac {2}{3}, \gamma = \frac {2}{3}$

\item $l \approx m^{\alpha}g^{\beta}\hbar^{\gamma}$
$[m] = [kg]^{\alpha}[\frac {m}{s^2}]^{\beta}[\frac {kgm^2}{s}]^{\gamma}$

so set up equations of exponents

$kg  \rightarrow 0 = 1\alpha + 0\alpha + 1\alpha$\\
$m  \rightarrow 1 = 0\beta + 1\beta + 2\beta$\\
$sec  \rightarrow 0 = 1\gamma + 0\gamma + 1\gamma$\\

solve simultaneous equations and this gives 
$\alpha = \frac {2}{3}, \beta = -\frac {1}{3}, \gamma = \frac {2}{3}$

repeat for t $\rightarrow$
$\alpha = -\frac {1}{3}, \beta = -\frac {2}{3}, \gamma = \frac {1}{3}$

repeat for v $\rightarrow$
$\alpha = -\frac {1}{3}, \beta = \frac {1}{3}, \gamma = \frac {1}{3}$\\

\item If E= 0 requires V=0 and x=0, then $\delta x \delta p = 0$ which violates the uncertainty principle (note p = mv)

\item from above using parameters for mass of electron and g
\begin{center}
$E \approx (mg^2\hbar^2)^{1/3} = (9.11*10^{-31}*(9.8)^2*(1.054*10^{-34})^2)^{1/3}$
$ =  9.9*10^{-33} J = 6.18*10^{-14}eV$
\end{center}

note that the ground state energy for hydrogen is 13.6eV

as $\hbar\rightarrow 0$ then $E\rightarrow 0$ as well.  As stated above E can never be exactly zero by the uncertainty principle so no this does not hold up.

\item Substituting into the approximations above gives

\begin{itemize}
  \item [] $E = 1*10^{-12}$ eV
  \item [] $l = 10*10^{-6}$ m
  \item [] $t= 1*10^{-3}$ seconds 
  \item [] $v = 0.01 \frac {m}{s}$
\end{itemize}
\end{enumerate}
\noindent\makebox[\linewidth]{\rule{\paperwidth}{0.4pt}}
\item \mbox{}\\
\noindent
\begin{enumerate}[i.]
\item using same process as above $L_P =  \sqrt( \frac {\hbar G} {c^3} )$\\
\item substituting in values to the last equations gives $L_P = 1.616*10^{-35}$ m\\
\item using same process as above $M_P =  \sqrt( \frac {\hbar c} {G} ) = 2.18*10^{-8}$ kg\\

length is much less than typical nuclear sizes and nuclear masses are much less than $M_P$ so no this is not a factor in nuclear physics.

(anyone have any better ideas??)
\end{enumerate}
\end{enumerate}
\noindent\makebox[\linewidth]{\rule{\paperwidth}{0.4pt}}

\end{document}