\documentclass[ps1.tex]{subfiles}
%\usepackage{enumerate}
%\usepackage[left=3cm,right=3cm,bottom=3cm,top=2cm]{geometry}
\begin{document}

\section* {(20 points) de Broglie Relations and the Scale of Quantum Effects}

\begin{enumerate}[(a)]
\item \underline{Light Waves as Particles}
The Photoelectric effect suggests that light of frequency $\nu$ can be regarded as consisting of photons of energy $E = h\nu$, where $h = 6.63\cdot10^{-27}$ erg*s.
\begin{enumerate}[i.]
\item Visible light has a wavelength in the range of 400-700 nm. What are the energy and frequency of a photon of visible light?
\item The microwave in my kitchen operates at roughly 2.5 GHz at a max power of $7.5\cdot10^9 \frac {erg}{s}$. How many photons per second can it emit? What about a low-power laser ($10^4 \frac{erg}{s}$ at 633 nm), or a cell phone ($4\cdot10^6 \frac{erg}{s}$ at 850 MHz)?
\item How many such microwave photons does it take to warm a 200ml glass of water by $10C$? (The heat capacity of water is roughly $4.18\cdot10^7 \frac {erg}{gK})$
\item At a given power of an electromagnetic wave, do you expect a classical wave
description to work better for radio frequencies, or for X-rays?
\end{enumerate}
\item \underline{Matter Particles as Waves}
If a wavelength can be associated with every moving particle, then why are we not forcibly made aware of this property in our everyday experience? In answering, calculate the de Broglie wavelength $\lambda = hp$ of each of the following particles

$p = mv$ and $\lambda = \frac {h}{p} = \frac {h}{mv}$

\begin{enumerate}[i.]
\item an automobile of mass 2 metric tons (2000 kg) traveling at a speed of 50 mph (22 ms )
\item a marble of mass 10 g moving with a speed of $10 \frac {cm}{s}$
\item a smoke particle of diameter $10^{-5}$ cm and a density of, say, (.2 $\frac {g}{cm^3}$  ) being
jostled about by air molecules at room temperature (T=300K) (assume that
the particle has the same translational kinetic energy as the thermal average
of the air molecules, $KE = \frac {3}{2} k_B T$ , with $k_B = 1.38\cdot10^{-16} \frac {erg}{k}$)
\item An $^{87}Rb$ atom that has been laser cooled to a temperature of $T = 100 \mu K$. Again, assume $KE = \frac {3}{2} k_B T$
\end{enumerate}
\end{enumerate}

\noindent\makebox[\linewidth]{\rule{\paperwidth}{0.4pt}}

\begin{enumerate}[(a)]
\item \mbox{} \\ 
\noindent
\begin{enumerate}[i.]
\item $\nu=\frac {c}{\lambda} \rightarrow E = h\nu = \frac{h\cdot c}{\lambda}$

for $\lambda = 400$ nm then \\
$\nu = \frac {3\cdot 10^8}{400\cdot 10^{-9}} = 750$  THz and \\
$E=\frac{(6.63\cdot 10^{-27})*(3\cdot 10^8)}{400\cdot 10^{-9}} \approx = 5\cdot 10^{-12}$ ergs\\

for $\lambda = 700$ nm then \\
$\nu = \frac {3\cdot 10^8}{700\cdot 10^{-9}} = 430$  THz and \\
$E=\frac{(6.63\cdot 10^{-27})*(3\cdot 10^8)}{700\cdot 10^{-9}} \approx = 2.9\cdot 10^{-12}$ ergs

\item $E= (6.626\cdot 10^{-34})\cdot(2.5\cdot10^9) = 1.66\cdot 10^{-24}$ Joules/photon\\
$N = \frac{750}{1.66\cdot 10^{-24}} = 4.5\cdot10^{26}$ photons/sec\\

$\lambda = 644$ nm then $E= \frac{h\cdot c}{\lambda} = \frac {6.626\cdot10^{-34}\cdot 2.988\cdot10^8}{633\cdot10^{-9}} = 2.96\cdot10^{-19}$ J. and\\
$N_{laser} = \frac {1\cdot10^-3}{2.96\cdot10^{-19}} = 3.38\cdot10^{15}$ photons/s. \\

Cell phone $\nu = 850\cdot10^6$ Hz\\
$E= (6.626\cdot 10^{-34})\cdot(850\cdot10^6) = 5.63\cdot 10^{-25}$ Joules/photon\\
$N = \frac{0.4}{5.63\cdot 10^{-25}} = 7.10\cdot10^{23}$ photons/s\\

\item 1 liter water weights 1kg so or in this problem 0.2kg

$E = Cm \Delta T = 4.18\cdot10^3*0.2*10 = 8.36\cdot10^3$ Joules

$N = \frac {8.36\cdot10^3}{1.62\cdot10^{-24}} = 5.16\cdot10^{27}$ photons.\\

\item (My answer)
Classical Theory works best for radio frequencies as the number of photons is larger for radio frequencies by about $10^{11}$ with large numbers of photons classical statistics describes behavior very well with low numbers quantum effects become larger.

(MIT OCW Solutions answer)
We expect a classical wave description to work better for radiofrequencies. The classical electromagnetic description of photons works fine when a number of photons is large. However, this description breaks down when we try to describe a single photon. At a given power of an electromagnetic wave, a number of photons for radiofrequencies in the detection window is much larger than that for X-rays. Therefore, the wave description of photons is adequate for radiofrequencies but is not adequate for X-rays. We can estimate a ratio of a number of photons for radiofrequencies and for X-rays at a given power (i.e. 1 W). Energy of radiofrequencies is about $10^{-7}$ eV and that of X-rays is about $10^4$ eV.

$R = \frac {N_{radio}}{N_{x-ray}} = \frac {10^4}{10^{-7}} = 10^{11}$.

Therefore, at a given power, for every X-ray photon, there are about $10^{11}$ radiofrequency photons.
Assume that a relaxation time of a photon detector is about 1 ps ($10^{-12}$ s). Our detector can detect a single photon if it arrives at the detector at a rate of 1 photon per 1 ps. This time scale determines our detection-window time scale. Therefore, the power of the He-Ne laser in which we expect quantum effect to become important is:

$P = \frac {E_{photon}}{1 ps} = \frac {2.96\cdot10^{-19}}{10^{-12}} = 2.96\cdot10^{-7}$ W.

\end{enumerate}

\noindent\makebox[\linewidth]{\rule{\paperwidth}{0.4pt}}


\item  \mbox{} \\ 
\noindent
\begin{enumerate}[i.]
\item $\lambda = \frac {6.626\cdot10^{-34}}{2\cdot10^3\cdot 22} = 1.5\cdot10^{-38}$

\item $\lambda = \frac {6.626\cdot10^{-34}}{0.01\cdot .1} = 6.6\cdot10^{-31}$

\item Volume of the particle is $V=\frac {4}{3}\pi r^3 = \frac {4}{3}\pi(1\cdot10^{-7})^3 = 4.2\cdot10^{-21}m^3$\\
the mass is $m=V\cdot \rho = (4.2\cdot 10^{-21}m^3)\cdot (200 \frac {kg}{m^3}) = 8.4\cdot10^{-19}$

given that $E_{KE} = \frac {3}{2}k_BT \rightarrow E_{KE} = \frac {3}{2}\cdot(1.38\cdot10^{-23})(300) = 6.24\cdot10^{-21}$
so $\frac{1}{2}mv^2 =  6.24\cdot10^{-21} \rightarrow v = \sqrt(\frac {2\cdot(6.24\cdot10^{-21})}{8.4\cdot10^{-19}}) = .122 \frac {m}{s}$\\
from this $p = (8.4\cdot10^{-19})\cdot(.122) = 1.02\cdot10^{-19}$ and then $\lambda = \frac {h}{p} = \frac {6.602\cdot10^{-34}}{1.02\cdot10^{-19}} = 5.9\cdot10^{-15}$ m\\

\item $m = 1.44\cdot10^{-25}$ kg\\
$E_{KE} = \frac {3}{2}k_BT \rightarrow E_{KE} = \frac {3}{2}\cdot(1.38\cdot10^{-23})(100\cdot10^{-6}) = 2.07\cdot 10^{-27}$
so $\frac{1}{2}mv^2 = 2.07\cdot 10^{-27} \rightarrow v = \sqrt(\frac {2(2.07\cdot 10^{-27})}{1.44\cdot10^{-25}} = .17$ m\\
from this $p = (1.44\cdot10^{-25})\cdot(.17) = 2.44\cdot10^{-26}$ and then \\
$\lambda = \frac {h}{p} = \frac {6.602\cdot10^{-34}} {2.44\cdot10^{-26}} = 2.7\cdot10{-8}$\\

\end{enumerate}
\end{enumerate}

\noindent\makebox[\linewidth]{\rule{\paperwidth}{0.4pt}}

\end{document}